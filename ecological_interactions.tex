% Options for packages loaded elsewhere
\PassOptionsToPackage{unicode}{hyperref}
\PassOptionsToPackage{hyphens}{url}
\PassOptionsToPackage{dvipsnames,svgnames,x11names}{xcolor}
%
\documentclass[
  letterpaper,
  DIV=11,
  numbers=noendperiod]{scrartcl}

\usepackage{amsmath,amssymb}
\usepackage{lmodern}
\usepackage{iftex}
\ifPDFTeX
  \usepackage[T1]{fontenc}
  \usepackage[utf8]{inputenc}
  \usepackage{textcomp} % provide euro and other symbols
\else % if luatex or xetex
  \usepackage{unicode-math}
  \defaultfontfeatures{Scale=MatchLowercase}
  \defaultfontfeatures[\rmfamily]{Ligatures=TeX,Scale=1}
\fi
% Use upquote if available, for straight quotes in verbatim environments
\IfFileExists{upquote.sty}{\usepackage{upquote}}{}
\IfFileExists{microtype.sty}{% use microtype if available
  \usepackage[]{microtype}
  \UseMicrotypeSet[protrusion]{basicmath} % disable protrusion for tt fonts
}{}
\makeatletter
\@ifundefined{KOMAClassName}{% if non-KOMA class
  \IfFileExists{parskip.sty}{%
    \usepackage{parskip}
  }{% else
    \setlength{\parindent}{0pt}
    \setlength{\parskip}{6pt plus 2pt minus 1pt}}
}{% if KOMA class
  \KOMAoptions{parskip=half}}
\makeatother
\usepackage{xcolor}
\setlength{\emergencystretch}{3em} % prevent overfull lines
\setcounter{secnumdepth}{5}
% Make \paragraph and \subparagraph free-standing
\ifx\paragraph\undefined\else
  \let\oldparagraph\paragraph
  \renewcommand{\paragraph}[1]{\oldparagraph{#1}\mbox{}}
\fi
\ifx\subparagraph\undefined\else
  \let\oldsubparagraph\subparagraph
  \renewcommand{\subparagraph}[1]{\oldsubparagraph{#1}\mbox{}}
\fi

\usepackage{color}
\usepackage{fancyvrb}
\newcommand{\VerbBar}{|}
\newcommand{\VERB}{\Verb[commandchars=\\\{\}]}
\DefineVerbatimEnvironment{Highlighting}{Verbatim}{commandchars=\\\{\}}
% Add ',fontsize=\small' for more characters per line
\usepackage{framed}
\definecolor{shadecolor}{RGB}{241,243,245}
\newenvironment{Shaded}{\begin{snugshade}}{\end{snugshade}}
\newcommand{\AlertTok}[1]{\textcolor[rgb]{0.68,0.00,0.00}{#1}}
\newcommand{\AnnotationTok}[1]{\textcolor[rgb]{0.37,0.37,0.37}{#1}}
\newcommand{\AttributeTok}[1]{\textcolor[rgb]{0.40,0.45,0.13}{#1}}
\newcommand{\BaseNTok}[1]{\textcolor[rgb]{0.68,0.00,0.00}{#1}}
\newcommand{\BuiltInTok}[1]{\textcolor[rgb]{0.00,0.23,0.31}{#1}}
\newcommand{\CharTok}[1]{\textcolor[rgb]{0.13,0.47,0.30}{#1}}
\newcommand{\CommentTok}[1]{\textcolor[rgb]{0.37,0.37,0.37}{#1}}
\newcommand{\CommentVarTok}[1]{\textcolor[rgb]{0.37,0.37,0.37}{\textit{#1}}}
\newcommand{\ConstantTok}[1]{\textcolor[rgb]{0.56,0.35,0.01}{#1}}
\newcommand{\ControlFlowTok}[1]{\textcolor[rgb]{0.00,0.23,0.31}{#1}}
\newcommand{\DataTypeTok}[1]{\textcolor[rgb]{0.68,0.00,0.00}{#1}}
\newcommand{\DecValTok}[1]{\textcolor[rgb]{0.68,0.00,0.00}{#1}}
\newcommand{\DocumentationTok}[1]{\textcolor[rgb]{0.37,0.37,0.37}{\textit{#1}}}
\newcommand{\ErrorTok}[1]{\textcolor[rgb]{0.68,0.00,0.00}{#1}}
\newcommand{\ExtensionTok}[1]{\textcolor[rgb]{0.00,0.23,0.31}{#1}}
\newcommand{\FloatTok}[1]{\textcolor[rgb]{0.68,0.00,0.00}{#1}}
\newcommand{\FunctionTok}[1]{\textcolor[rgb]{0.28,0.35,0.67}{#1}}
\newcommand{\ImportTok}[1]{\textcolor[rgb]{0.00,0.46,0.62}{#1}}
\newcommand{\InformationTok}[1]{\textcolor[rgb]{0.37,0.37,0.37}{#1}}
\newcommand{\KeywordTok}[1]{\textcolor[rgb]{0.00,0.23,0.31}{#1}}
\newcommand{\NormalTok}[1]{\textcolor[rgb]{0.00,0.23,0.31}{#1}}
\newcommand{\OperatorTok}[1]{\textcolor[rgb]{0.37,0.37,0.37}{#1}}
\newcommand{\OtherTok}[1]{\textcolor[rgb]{0.00,0.23,0.31}{#1}}
\newcommand{\PreprocessorTok}[1]{\textcolor[rgb]{0.68,0.00,0.00}{#1}}
\newcommand{\RegionMarkerTok}[1]{\textcolor[rgb]{0.00,0.23,0.31}{#1}}
\newcommand{\SpecialCharTok}[1]{\textcolor[rgb]{0.37,0.37,0.37}{#1}}
\newcommand{\SpecialStringTok}[1]{\textcolor[rgb]{0.13,0.47,0.30}{#1}}
\newcommand{\StringTok}[1]{\textcolor[rgb]{0.13,0.47,0.30}{#1}}
\newcommand{\VariableTok}[1]{\textcolor[rgb]{0.07,0.07,0.07}{#1}}
\newcommand{\VerbatimStringTok}[1]{\textcolor[rgb]{0.13,0.47,0.30}{#1}}
\newcommand{\WarningTok}[1]{\textcolor[rgb]{0.37,0.37,0.37}{\textit{#1}}}

\providecommand{\tightlist}{%
  \setlength{\itemsep}{0pt}\setlength{\parskip}{0pt}}\usepackage{longtable,booktabs,array}
\usepackage{calc} % for calculating minipage widths
% Correct order of tables after \paragraph or \subparagraph
\usepackage{etoolbox}
\makeatletter
\patchcmd\longtable{\par}{\if@noskipsec\mbox{}\fi\par}{}{}
\makeatother
% Allow footnotes in longtable head/foot
\IfFileExists{footnotehyper.sty}{\usepackage{footnotehyper}}{\usepackage{footnote}}
\makesavenoteenv{longtable}
\usepackage{graphicx}
\makeatletter
\def\maxwidth{\ifdim\Gin@nat@width>\linewidth\linewidth\else\Gin@nat@width\fi}
\def\maxheight{\ifdim\Gin@nat@height>\textheight\textheight\else\Gin@nat@height\fi}
\makeatother
% Scale images if necessary, so that they will not overflow the page
% margins by default, and it is still possible to overwrite the defaults
% using explicit options in \includegraphics[width, height, ...]{}
\setkeys{Gin}{width=\maxwidth,height=\maxheight,keepaspectratio}
% Set default figure placement to htbp
\makeatletter
\def\fps@figure{htbp}
\makeatother

\KOMAoption{captions}{tableheading}
\makeatletter
\makeatother
\makeatletter
\makeatother
\makeatletter
\@ifpackageloaded{caption}{}{\usepackage{caption}}
\AtBeginDocument{%
\ifdefined\contentsname
  \renewcommand*\contentsname{Table of contents}
\else
  \newcommand\contentsname{Table of contents}
\fi
\ifdefined\listfigurename
  \renewcommand*\listfigurename{List of Figures}
\else
  \newcommand\listfigurename{List of Figures}
\fi
\ifdefined\listtablename
  \renewcommand*\listtablename{List of Tables}
\else
  \newcommand\listtablename{List of Tables}
\fi
\ifdefined\figurename
  \renewcommand*\figurename{Figure}
\else
  \newcommand\figurename{Figure}
\fi
\ifdefined\tablename
  \renewcommand*\tablename{Table}
\else
  \newcommand\tablename{Table}
\fi
}
\@ifpackageloaded{float}{}{\usepackage{float}}
\floatstyle{ruled}
\@ifundefined{c@chapter}{\newfloat{codelisting}{h}{lop}}{\newfloat{codelisting}{h}{lop}[chapter]}
\floatname{codelisting}{Listing}
\newcommand*\listoflistings{\listof{codelisting}{List of Listings}}
\makeatother
\makeatletter
\@ifpackageloaded{caption}{}{\usepackage{caption}}
\@ifpackageloaded{subcaption}{}{\usepackage{subcaption}}
\makeatother
\makeatletter
\@ifpackageloaded{tcolorbox}{}{\usepackage[many]{tcolorbox}}
\makeatother
\makeatletter
\@ifundefined{shadecolor}{\definecolor{shadecolor}{rgb}{.97, .97, .97}}
\makeatother
\makeatletter
\makeatother
\ifLuaTeX
  \usepackage{selnolig}  % disable illegal ligatures
\fi
\IfFileExists{bookmark.sty}{\usepackage{bookmark}}{\usepackage{hyperref}}
\IfFileExists{xurl.sty}{\usepackage{xurl}}{} % add URL line breaks if available
\urlstyle{same} % disable monospaced font for URLs
\hypersetup{
  colorlinks=true,
  linkcolor={blue},
  filecolor={Maroon},
  citecolor={Blue},
  urlcolor={Blue},
  pdfcreator={LaTeX via pandoc}}

\author{}
\date{}

\begin{document}
\ifdefined\Shaded\renewenvironment{Shaded}{\begin{tcolorbox}[enhanced, sharp corners, frame hidden, boxrule=0pt, interior hidden, breakable, borderline west={3pt}{0pt}{shadecolor}]}{\end{tcolorbox}}\fi

\renewcommand*\contentsname{Table of contents}
{
\hypersetup{linkcolor=}
\setcounter{tocdepth}{3}
\tableofcontents
}
\hypertarget{ecological-interactions-dynamics}{%
\section{Ecological Interactions \&
Dynamics}\label{ecological-interactions-dynamics}}

Principles in Fisheries Science

Max Lindmark

November 3, 2022

Swedish University of Agricultural Sciences

\hypertarget{who-am-i}{%
\subsection{Who am I?}\label{who-am-i}}

\begin{itemize}
\item
  MSc in Applied Marine and Fisheries Ecology, Aberdeen UK
\item
  PhD in Marine Ecology, theoretical population and community ecology
\item
  Postdoc at SLU on spatiotemporal aspects of cod-flounder interactions
\item
  \href{mailto:max.lindmark@slu.se}{\nolinkurl{max.lindmark@slu.se}}
\end{itemize}

\hypertarget{who-are-you}{%
\subsection{Who are you \ldots{} ?}\label{who-are-you}}

\hypertarget{plan-for-today}{%
\subsection{Plan for today}\label{plan-for-today}}

\begin{itemize}
\item
  Theoretical

  \begin{itemize}
  \item
    1: Brief recap on population ecology
  \item
    2: Species interactions!
  \end{itemize}
\item
  ``Empirical''

  \begin{itemize}
  \item
    3: Quantifying species interactions in natural systems
  \item
    4: Species interactions in fisheries ecology
  \end{itemize}
\end{itemize}

theoretical = mathematical

\hypertarget{plan-for-today-1}{%
\subsection{Plan for today}\label{plan-for-today-1}}

\begin{itemize}
\item
  Have questions?

  \begin{itemize}
  \tightlist
  \item
    Just interrupt and ask, write in chat or ask during our break.
    Thanks!
  \end{itemize}
\item
  Slides can be found here: \url{https://github.com/maxlindmark/pfs}
\item
  And on Canvas
\end{itemize}

\hypertarget{motivating-questions}{%
\subsection{Motivating questions}\label{motivating-questions}}

\begin{itemize}
\item
  We want to know what makes fish abundance vary through time and
  space\ldots{}

  \begin{itemize}
  \item
    For conservation
  \item
    For sustainable management of marine resources
  \item
    Because it is a basic scientific question (ecology is a young
    discipline!)
  \end{itemize}
\end{itemize}

\hypertarget{brief-recap-on-population-ecology}{%
\section{1. Brief recap on population
ecology}\label{brief-recap-on-population-ecology}}

\hypertarget{recap}{%
\subsection{Recap}\label{recap}}

\begin{itemize}
\item
  \textbf{Population ecology}: how and why animal populations change in
  numbers over time and space
\item
  \textbf{Community ecology}: how and why species abundance,
  composition, diversity and structure change over time and space
\item
  \textbf{Food web ecology}: energy transfers (feeding links) between
  species in food webs
\end{itemize}

\hypertarget{population-ecology}{%
\subsection{Population ecology}\label{population-ecology}}

The branch of science that attempts to describe, understand, and predict
the growth (positive and negative) of animal and plant populations

\hypertarget{why-do-we-use-models}{%
\subsection{Why do we use models?}\label{why-do-we-use-models}}

\begin{itemize}
\item
  Capture the essence of a process so that we can generalize across
  systems
\item
  Simplest representation of a population:
\end{itemize}

\begin{Shaded}
\begin{Highlighting}[]
\NormalTok{d }\OtherTok{\textless{}{-}} \FunctionTok{data.frame}\NormalTok{(}\AttributeTok{y =} \DecValTok{1}\NormalTok{, }\AttributeTok{x =} \DecValTok{1}\NormalTok{, }\AttributeTok{family =} \StringTok{"mono"}\NormalTok{)}

\FunctionTok{ggplot}\NormalTok{(d, }\FunctionTok{aes}\NormalTok{(x, y)) }\SpecialCharTok{+}
  \FunctionTok{geom\_point}\NormalTok{(}\AttributeTok{shape =} \DecValTok{21}\NormalTok{, }\AttributeTok{size =} \DecValTok{40}\NormalTok{, }\AttributeTok{stroke =} \DecValTok{2}\NormalTok{) }\SpecialCharTok{+}
  \FunctionTok{theme\_void}\NormalTok{()}
\end{Highlighting}
\end{Shaded}

\begin{figure}[H]

{\centering \includegraphics{ecological_interactions_files/figure-pdf/unnamed-chunk-1-1.pdf}

}

\end{figure}

\begin{itemize}
\tightlist
\item
  What do we assume in this model?
\end{itemize}

\hypertarget{the-simplest-population-model}{%
\subsection{The simplest population
model}\label{the-simplest-population-model}}

Population size (numbers) at time \(t\): \[N(t)\]

Rate of change: \[\frac{dN}{dt}\]

Per capita rate of change: \[\frac{1}{N}\frac{dN}{dt}\]

\hypertarget{the-simplest-population-model-1}{%
\subsection{The simplest population
model}\label{the-simplest-population-model-1}}

\[r = b - d\]

\[\frac{dN}{dt}=rN\]

\hypertarget{the-simplest-population-model-2}{%
\subsection{The simplest population
model}\label{the-simplest-population-model-2}}

\begin{itemize}
\tightlist
\item
  Solution to this first-order differential equation:
\end{itemize}

\[N(t)=N_0e^{rt}\]

\begin{itemize}
\tightlist
\item
  Exponential growth for positive \(r\) and exponential decline for
  negative \(r\)
\end{itemize}

\hypertarget{the-simplest-population-model-3}{%
\subsection{The simplest population
model}\label{the-simplest-population-model-3}}

\[N(t)=N_0e^{rt}\]

\begin{Shaded}
\begin{Highlighting}[]
\NormalTok{n }\OtherTok{\textless{}{-}} \DecValTok{1}\SpecialCharTok{:}\DecValTok{50}
\NormalTok{r }\OtherTok{\textless{}{-}} \FloatTok{0.1}
\NormalTok{d }\OtherTok{\textless{}{-}} \FunctionTok{data.frame}\NormalTok{(}\AttributeTok{N =}\NormalTok{ n) }\SpecialCharTok{\%\textgreater{}\%} 
  \FunctionTok{mutate}\NormalTok{(}\AttributeTok{dNdt =}\NormalTok{ r}\SpecialCharTok{*}\NormalTok{N,}
         \AttributeTok{pc\_dNdt =}\NormalTok{ (r}\SpecialCharTok{*}\NormalTok{N)}\SpecialCharTok{/}\NormalTok{N)}
\end{Highlighting}
\end{Shaded}

\hypertarget{the-simplest-population-model-4}{%
\subsection{The simplest population
model}\label{the-simplest-population-model-4}}

\begin{figure}

{\centering \includegraphics{ecological_interactions_files/figure-pdf/unnamed-chunk-3-1.pdf}

}

\end{figure}

\begin{itemize}
\tightlist
\item
  Per-capita growth rates unaffected by density
\item
  Population growth rates increase with size!
\end{itemize}

\hypertarget{the-simplest-population-model-5}{%
\subsection{The simplest population
model}\label{the-simplest-population-model-5}}

\begin{itemize}
\tightlist
\item
  Equilibrium?
\end{itemize}

\[\frac{dN}{dt}=rN\]

\hypertarget{the-simplest-population-model-6}{%
\subsection{The simplest population
model}\label{the-simplest-population-model-6}}

\begin{itemize}
\tightlist
\item
  What's wrong with this model?
\end{itemize}

\[\frac{dN}{dt}=rN\]

\hypertarget{population-regulation}{%
\subsection{Population regulation}\label{population-regulation}}

\begin{itemize}
\item
  Density or density-independent factors?
\item
  Strongly debated in the 1950's
\item
  Any ideas for the simplest form of density dependence?
\end{itemize}

\hypertarget{population-regulation-1}{%
\subsection{Population regulation}\label{population-regulation-1}}

\begin{Shaded}
\begin{Highlighting}[]
\NormalTok{p1 }\SpecialCharTok{+} \FunctionTok{theme}\NormalTok{(}\AttributeTok{aspect.ratio =} \DecValTok{1}\SpecialCharTok{/}\DecValTok{2}\NormalTok{)}
\end{Highlighting}
\end{Shaded}

\begin{figure}[H]

{\centering \includegraphics{ecological_interactions_files/figure-pdf/unnamed-chunk-4-1.pdf}

}

\end{figure}

\begin{center}\rule{0.5\linewidth}{0.5pt}\end{center}

\hypertarget{the-logistic-growth-model}{%
\subsection{The logistic growth model}\label{the-logistic-growth-model}}

\textbf{John Graunt (1662)}

\begin{itemize}
\item
  Studied lists of births and deaths (demography) in London
\item
  Adam and Eve: 6000 years ago\ldots{} doubling time is 64 years\ldots{}
\item
  Should be \emph{far} more people now than there is
\end{itemize}

\begin{figure}

\hfill{} \includegraphics[width=\textwidth,height=2.60417in]{images/JohnGraunt.png}

\end{figure}

\hypertarget{the-logistic-growth-model-1}{%
\subsection{The logistic growth
model}\label{the-logistic-growth-model-1}}

\textbf{Sir William Petty (1683)}

\begin{itemize}
\item
  Fool, you haven't accounted for the biblical flood!
\item
  Reduce doubling time, change start time
\end{itemize}

\begin{figure}

\hfill{} \includegraphics{images/sir_will.png}

\end{figure}

\begin{center}\rule{0.5\linewidth}{0.5pt}\end{center}

\begin{figure}

{\centering \includegraphics{images/close_will.png}

}

\end{figure}

\begin{center}\rule{0.5\linewidth}{0.5pt}\end{center}

\begin{figure}

{\centering \includegraphics{images/closest_will.png}

}

\end{figure}

\hypertarget{the-logistic-growth-model-2}{%
\subsection{The logistic growth
model}\label{the-logistic-growth-model-2}}

\textbf{Reverend Thomas Robert Malthus (1798)}

\begin{itemize}
\item
  \emph{An Essay on the Principle of Populations}

  \begin{itemize}
  \tightlist
  \item
    If population grows faster than supply\ldots{}
  \item
    \ldots{} much human misery!
  \end{itemize}
\end{itemize}

\begin{figure}

\hfill{} \includegraphics{images/TomasMaltus.png}

\end{figure}

\hypertarget{the-logistic-growth-model-3}{%
\subsection{The logistic growth
model}\label{the-logistic-growth-model-3}}

\textbf{Pierre-Francois Verhulst (1845)}

\begin{itemize}
\tightlist
\item
  Derived the logistic equation
\end{itemize}

\textbf{Raymond Pearl} and \textbf{Lowell Reed (1920)}

\begin{itemize}
\tightlist
\item
  Re-discovered it: Law of nature?
\end{itemize}

\begin{figure}

\hfill{} \includegraphics[width=\textwidth,height=2.60417in]{images/verhulst.png}

\end{figure}

\hypertarget{the-logistic-growth-model-4}{%
\subsection{The logistic growth
model}\label{the-logistic-growth-model-4}}

\[\frac{dN}{dt}=rN(1-\frac{N}{K})\]

\begin{Shaded}
\begin{Highlighting}[]
\NormalTok{n }\OtherTok{\textless{}{-}} \DecValTok{0}\SpecialCharTok{:}\DecValTok{50}
\NormalTok{K }\OtherTok{\textless{}{-}} \DecValTok{40}

\NormalTok{d }\OtherTok{\textless{}{-}} \FunctionTok{data.frame}\NormalTok{(}\AttributeTok{N =}\NormalTok{ n) }\SpecialCharTok{\%\textgreater{}\%} 
  \FunctionTok{mutate}\NormalTok{(}\AttributeTok{dNdt =}\NormalTok{ r}\SpecialCharTok{*}\NormalTok{N}\SpecialCharTok{*}\NormalTok{(}\DecValTok{1} \SpecialCharTok{{-}}\NormalTok{ N}\SpecialCharTok{/}\NormalTok{K),}
         \AttributeTok{pc\_dNdt =}\NormalTok{ dNdt}\SpecialCharTok{/}\NormalTok{N)}
\end{Highlighting}
\end{Shaded}

\hypertarget{the-logistic-growth-model-5}{%
\subsection{The logistic growth
model}\label{the-logistic-growth-model-5}}

\begin{verbatim}
Warning: Removed 1 row(s) containing missing values (geom_path).
\end{verbatim}

\begin{figure}

{\centering \includegraphics{ecological_interactions_files/figure-pdf/unnamed-chunk-6-1.pdf}

}

\end{figure}

\hypertarget{the-logistic-growth-model-6}{%
\subsection{The logistic growth
model}\label{the-logistic-growth-model-6}}

\[\frac{dN}{dt}=rN(1-\frac{N}{K})\]

\begin{itemize}
\item
  How does the population grow?
\item
  Can solve this\ldots{} but also iteratively fill in \(N\)s'
\end{itemize}

\begin{Shaded}
\begin{Highlighting}[]
\NormalTok{r }\OtherTok{\textless{}{-}} \FloatTok{0.1}\NormalTok{; K }\OtherTok{\textless{}{-}} \DecValTok{40}\NormalTok{; N\_ini }\OtherTok{\textless{}{-}} \DecValTok{2}\NormalTok{; t }\OtherTok{\textless{}{-}} \DecValTok{100}
\NormalTok{N }\OtherTok{\textless{}{-}} \FunctionTok{rep}\NormalTok{(}\ConstantTok{NA}\NormalTok{, t)}
\NormalTok{N[}\DecValTok{1}\NormalTok{] }\OtherTok{\textless{}{-}}\NormalTok{ N\_ini}

\ControlFlowTok{for}\NormalTok{ (i }\ControlFlowTok{in} \DecValTok{2}\SpecialCharTok{:}\NormalTok{t) \{}
\NormalTok{N[i] }\OtherTok{\textless{}{-}}\NormalTok{ r }\SpecialCharTok{*}\NormalTok{ N[i}\DecValTok{{-}1}\NormalTok{] }\SpecialCharTok{*}\NormalTok{ (}\DecValTok{1}\SpecialCharTok{{-}}\NormalTok{(N[i }\SpecialCharTok{{-}} \DecValTok{1}\NormalTok{]}\SpecialCharTok{/}\NormalTok{K)) }\SpecialCharTok{+}\NormalTok{ N[i}\DecValTok{{-}1}\NormalTok{]}
\NormalTok{\}}
\end{Highlighting}
\end{Shaded}

\hypertarget{the-logistic-growth-model-7}{%
\subsection{The logistic growth
model}\label{the-logistic-growth-model-7}}

\begin{Shaded}
\begin{Highlighting}[]
\NormalTok{d }\OtherTok{\textless{}{-}} \FunctionTok{data.frame}\NormalTok{(}\AttributeTok{N =}\NormalTok{ N, }\AttributeTok{t =} \DecValTok{1}\SpecialCharTok{:}\NormalTok{t)}

\FunctionTok{ggplot}\NormalTok{(d, }\FunctionTok{aes}\NormalTok{(t, N)) }\SpecialCharTok{+}
  \FunctionTok{geom\_line}\NormalTok{(}\AttributeTok{size =} \FloatTok{1.2}\NormalTok{) }\SpecialCharTok{+} 
  \FunctionTok{ggtitle}\NormalTok{(}\StringTok{"Logistic growth curve"}\NormalTok{, }\AttributeTok{subtitle =} \StringTok{"r = 0.1 and K = 40"}\NormalTok{)}
\end{Highlighting}
\end{Shaded}

\begin{figure}[H]

{\centering \includegraphics{ecological_interactions_files/figure-pdf/unnamed-chunk-8-1.pdf}

}

\end{figure}

\hypertarget{the-logistic-growth-model-8}{%
\subsection{The logistic growth
model}\label{the-logistic-growth-model-8}}

\[\frac{dN}{dt}=rN(1-\frac{N}{K})\]

\begin{itemize}
\item
  Another useful technique is to find the equilibria with algebra
\item
  Set \(\frac{dN}{dt}=0\)
\end{itemize}

remind them of equilibria

\hypertarget{the-logistic-growth-model-9}{%
\subsection{The logistic growth
model}\label{the-logistic-growth-model-9}}

\[\frac{dN}{dt}=rN(1-\frac{N}{K})\]

\begin{itemize}
\item
  Another useful technique is to find the equilibria with algebra
\item
  set \(\frac{dN}{dt}=0\)
\item
  \(N_1^*=0\)
\item
  \(N_2^*=K\)
\end{itemize}

\hypertarget{the-logistic-growth-model-10}{%
\subsection{The logistic growth
model}\label{the-logistic-growth-model-10}}

\[\frac{dN}{dt}=rN(1-\frac{N}{K})\]

\begin{itemize}
\tightlist
\item
  One can also conduct stability analysis, but we will skip that
\end{itemize}

\hypertarget{the-logistic-growth-model-11}{%
\subsection{The logistic growth
model}\label{the-logistic-growth-model-11}}

\begin{itemize}
\tightlist
\item
  Is it really a law of nature?
\end{itemize}

\begin{Shaded}
\begin{Highlighting}[]
\NormalTok{p1 }\SpecialCharTok{+} \FunctionTok{theme}\NormalTok{(}\AttributeTok{aspect.ratio =} \DecValTok{1}\SpecialCharTok{/}\DecValTok{2}\NormalTok{)}
\end{Highlighting}
\end{Shaded}

\begin{verbatim}
Warning: Removed 1 row(s) containing missing values (geom_path).
\end{verbatim}

\begin{figure}[H]

{\centering \includegraphics{ecological_interactions_files/figure-pdf/unnamed-chunk-9-1.pdf}

}

\end{figure}

There are many other forms of dd\ldots{}

\hypertarget{the-logistic-growth-model-12}{%
\subsection{The logistic growth
model}\label{the-logistic-growth-model-12}}

\begin{itemize}
\tightlist
\item
  \ldots{} There are other models where growth depends on population
  size, e.g., Gompertz, and it may of course have different shape the
  linear!
\end{itemize}

\begin{Shaded}
\begin{Highlighting}[]
\NormalTok{p1 }\SpecialCharTok{+} \FunctionTok{theme}\NormalTok{(}\AttributeTok{aspect.ratio =} \DecValTok{1}\SpecialCharTok{/}\DecValTok{2}\NormalTok{)}
\end{Highlighting}
\end{Shaded}

\begin{verbatim}
Warning: Removed 1 row(s) containing missing values (geom_path).
\end{verbatim}

\begin{figure}[H]

{\centering \includegraphics{ecological_interactions_files/figure-pdf/unnamed-chunk-10-1.pdf}

}

\end{figure}

\hypertarget{species-interactions}{%
\section{2. Species interactions}\label{species-interactions}}

\hypertarget{species-interactions-1}{%
\subsection{2. Species interactions}\label{species-interactions-1}}

\begin{itemize}
\item
  \textbf{Predation{]}}{[}parasitism{]} (-, +)
\item
  \textbf{Competition} (-, -)
\item
  Mutualism (+, +)
\item
  Amensialism (0, -)
\item
  Commensalism (0, +)
\end{itemize}

always individuals interacting! Inter or intraspecific?

\hypertarget{predation}{%
\subsection{Predation}\label{predation}}

\begin{figure}

{\centering \includegraphics{images/big_fish.png}

}

\end{figure}

Staying in the theoretical lane a bit longer, we'll look at species
interactions with a set of \textbf{coupled} ODE's

\hypertarget{the-original-predator-prey-model}{%
\subsection{The original predator-prey
model}\label{the-original-predator-prey-model}}

\begin{itemize}
\item
  Lotka-Volterra
\item
  Published independently by Alfred J. Lotka and Vito Volterra in 1925
  \& 1926
\item
  Widely used and analyzed today
\end{itemize}

\hypertarget{the-original-predator-prey-model-1}{%
\subsection{The original predator-prey
model}\label{the-original-predator-prey-model-1}}

\begin{figure}

\begin{minipage}[t]{0.33\linewidth}

{\centering 

\raisebox{-\height}{

\includegraphics{images/lotka.png}

}

\caption{Alfred J. Lotka}

}

\end{minipage}%
%
\begin{minipage}[t]{0.33\linewidth}

{\centering 

\raisebox{-\height}{

\includegraphics{images/vito.png}

}

\caption{Vito Volterra}

}

\end{minipage}%

\end{figure}

\hypertarget{the-original-predator-prey-model-2}{%
\subsection{The original predator-prey
model}\label{the-original-predator-prey-model-2}}

\begin{figure}

\begin{minipage}[t]{0.33\linewidth}

{\centering 

\raisebox{-\height}{

\includegraphics{images/lotka.png}

}

\caption{Alfred J. Lotka}

}

\end{minipage}%
%
\begin{minipage}[t]{0.33\linewidth}

{\centering 

\raisebox{-\height}{

\includegraphics{images/vito.png}

}

\caption{Vito Volterra}

}

\end{minipage}%
%
\begin{minipage}[t]{0.33\linewidth}

{\centering 

\raisebox{-\height}{

\includegraphics{images/umberto.png}

}

\caption{Umberto D'Ancona}

}

\end{minipage}%
\newline
\begin{minipage}[t]{0.33\linewidth}

{\centering 

Volterra's enquiry was inspired through his interactions with the marine
biologist Umberto D'Ancona, who was courting his daughter Luisa Volterra
at the time and later was to become his son-in-law. D'Ancona studied the
fish catches in the Adriatic Sea and had noticed that the percentage of
Selachians caught had increased during the years of World War I
(1914â€``18), and the relative abundance of prey followed the opposite
pattern. This puzzled him, as the fishing effort had been very much
reduced during the war years.

}

\end{minipage}%

\end{figure}

\hypertarget{the-original-predator-prey-model-3}{%
\subsection{The original predator-prey
model}\label{the-original-predator-prey-model-3}}

\begin{figure}

\begin{minipage}[t]{0.50\linewidth}

{\centering 

\raisebox{-\height}{

\includegraphics{images/fish_market.png}

}

}

\end{minipage}%
%
\begin{minipage}[t]{0.50\linewidth}

{\centering 

\raisebox{-\height}{

\includegraphics{images/fish_market2.png}

}

}

\end{minipage}%

\end{figure}

\hypertarget{the-original-predator-prey-model-4}{%
\subsection{The original predator-prey
model}\label{the-original-predator-prey-model-4}}

\begin{itemize}
\tightlist
\item
  Volterra wrote down these eqns to Umberto's:
\end{itemize}

\[\frac{dN}{dt}=\alpha N - \beta NP\]
\[\frac{dP}{dt}=\delta \beta NP-\gamma P\]

\begin{itemize}
\tightlist
\item
  How does the prey grow in absence of predators? How does the predator
  population decline in the absence of prey? What is \(\beta\) and
  \(\delta\)?
\end{itemize}

\begin{enumerate}
\def\labelenumi{\arabic{enumi})}
\tightlist
\item
  exponentially (set P = 0 and it becomes the first growth model we
  looked at)
\item
  Also exponentially
\item
  Predation/attack rate and conversion efficiency
\end{enumerate}

\begin{itemize}
\tightlist
\item
  The cost to the victim is always greater than the benefit to the
  enemy.
\end{itemize}

\hypertarget{lotka-volterra-model}{%
\subsection{Lotka-Volterra model}\label{lotka-volterra-model}}

\begin{itemize}
\tightlist
\item
  Let's break it down
\end{itemize}

\[\frac{dN}{dt}=\alpha N - \beta NP\]

\[\frac{dP}{dt}=\delta \beta NP-\gamma P\]

\hypertarget{lotka-volterra-model-1}{%
\subsection{Lotka-Volterra model}\label{lotka-volterra-model-1}}

\begin{itemize}
\item
  Zero-net growth isoclines
\item
  set \(\frac{dN}{dt}\) or \(\frac{dP}{dt}\) = 0, break out \(P\) and
  \(N\)
\end{itemize}

\[\frac{dN}{dt}=\alpha N - \beta NP\]

\[\frac{dP}{dt}=\delta \beta NP-\gamma P\]

How do we find the isoclines? what do they mean here

\hypertarget{prey-zero-growth-isoclines}{%
\subsection{Prey zero growth
isoclines}\label{prey-zero-growth-isoclines}}

\[\frac{dN}{dt}=\alpha N - \beta NP\]

\[0=\alpha N - \beta NP\] \[0=\alpha - \beta P\]
\[P=\frac{\alpha}  {\beta} \]

\hypertarget{predator-zero-growth-isoclines}{%
\subsection{Predator zero growth
isoclines}\label{predator-zero-growth-isoclines}}

\[\frac{dP}{dt}=\delta \beta NP-\gamma P\]

\[0=\delta \beta NP-\gamma P\] \[0=\delta \beta N-\gamma\]

\[P=\frac{\gamma}  {\delta\beta} \]

This shows us that the prey increase whenever the predator abundance is
below a certain value. We found that value above, and it is P=r/a. When
does the predator increase or decrease? When the prey is above or below
a particular value, N=m/(ea). The only time these populations are at
rest is when both of these are true.

\hypertarget{lotka-volterra-model-2}{%
\subsection{Lotka-Volterra model}\label{lotka-volterra-model-2}}

\begin{Shaded}
\begin{Highlighting}[]
\CommentTok{\# https://strimas.com/post/lotka{-}volterra/}
\CommentTok{\# parameters}
\NormalTok{pars }\OtherTok{\textless{}{-}} \FunctionTok{c}\NormalTok{(}\AttributeTok{alpha =} \DecValTok{1}\NormalTok{, }\AttributeTok{beta =} \FloatTok{0.2}\NormalTok{, }\AttributeTok{delta =} \FloatTok{0.5}\NormalTok{, }\AttributeTok{gamma =} \FloatTok{0.2}\NormalTok{)}
\CommentTok{\# initial state }
\NormalTok{init }\OtherTok{\textless{}{-}} \FunctionTok{c}\NormalTok{(}\AttributeTok{x =} \DecValTok{1}\NormalTok{, }\AttributeTok{y =} \DecValTok{2}\NormalTok{)}
\CommentTok{\# times}
\NormalTok{times }\OtherTok{\textless{}{-}} \FunctionTok{seq}\NormalTok{(}\DecValTok{0}\NormalTok{, }\DecValTok{100}\NormalTok{, }\AttributeTok{by =} \DecValTok{1}\NormalTok{)}

\NormalTok{deriv }\OtherTok{\textless{}{-}} \ControlFlowTok{function}\NormalTok{(t, state, pars) \{}
  \FunctionTok{with}\NormalTok{(}\FunctionTok{as.list}\NormalTok{(}\FunctionTok{c}\NormalTok{(state, pars)), \{}
\NormalTok{    d\_x }\OtherTok{\textless{}{-}}\NormalTok{ alpha }\SpecialCharTok{*}\NormalTok{ x }\SpecialCharTok{{-}}\NormalTok{ beta }\SpecialCharTok{*}\NormalTok{ x }\SpecialCharTok{*}\NormalTok{ y}
\NormalTok{    d\_y }\OtherTok{\textless{}{-}}\NormalTok{ delta }\SpecialCharTok{*}\NormalTok{ beta }\SpecialCharTok{*}\NormalTok{ x }\SpecialCharTok{*}\NormalTok{ y }\SpecialCharTok{{-}}\NormalTok{ gamma }\SpecialCharTok{*}\NormalTok{ y}
    \FunctionTok{return}\NormalTok{(}\FunctionTok{list}\NormalTok{(}\FunctionTok{c}\NormalTok{(}\AttributeTok{x =}\NormalTok{ d\_x, }\AttributeTok{y =}\NormalTok{ d\_y)))}
\NormalTok{  \})}
\NormalTok{\}}

\NormalTok{lv\_results }\OtherTok{\textless{}{-}} \FunctionTok{ode}\NormalTok{(init, times, deriv, pars)}
\end{Highlighting}
\end{Shaded}

\hypertarget{lotka-volterra-model-3}{%
\subsection{Lotka-Volterra model}\label{lotka-volterra-model-3}}

\begin{itemize}
\tightlist
\item
  A cyclic model
\end{itemize}

\begin{Shaded}
\begin{Highlighting}[]
\NormalTok{lv\_results }\SpecialCharTok{\%\textgreater{}\%} 
  \FunctionTok{data.frame}\NormalTok{() }\SpecialCharTok{\%\textgreater{}\%} 
  \FunctionTok{gather}\NormalTok{(var, pop, }\SpecialCharTok{{-}}\NormalTok{time) }\SpecialCharTok{\%\textgreater{}\%} 
  \FunctionTok{mutate}\NormalTok{(}\AttributeTok{var =} \FunctionTok{if\_else}\NormalTok{(var }\SpecialCharTok{==} \StringTok{"x"}\NormalTok{, }\StringTok{"Prey"}\NormalTok{, }\StringTok{"Predator"}\NormalTok{)) }\SpecialCharTok{\%\textgreater{}\%} 
  \FunctionTok{ggplot}\NormalTok{(}\FunctionTok{aes}\NormalTok{(}\AttributeTok{x =}\NormalTok{ time, }\AttributeTok{y =}\NormalTok{ pop, }\AttributeTok{color =}\NormalTok{ var)) }\SpecialCharTok{+}
    \FunctionTok{geom\_line}\NormalTok{(}\AttributeTok{size =} \FloatTok{1.2}\NormalTok{) }\SpecialCharTok{+}
    \FunctionTok{scale\_color\_brewer}\NormalTok{(}\AttributeTok{palette =} \StringTok{"Set2"}\NormalTok{) }\SpecialCharTok{+}
    \FunctionTok{labs}\NormalTok{(}\AttributeTok{subtitle =} \FunctionTok{paste}\NormalTok{(}\FunctionTok{names}\NormalTok{(pars), pars, }\AttributeTok{sep =} \StringTok{" = "}\NormalTok{, }\AttributeTok{collapse =} \StringTok{"; "}\NormalTok{),}
         \AttributeTok{x =} \StringTok{"Time"}\NormalTok{, }\AttributeTok{y =} \StringTok{"Population density"}\NormalTok{) }\SpecialCharTok{+} 
    \FunctionTok{theme}\NormalTok{(}\AttributeTok{legend.position =} \StringTok{"bottom"}\NormalTok{)}
\end{Highlighting}
\end{Shaded}

\begin{figure}[H]

{\centering \includegraphics{ecological_interactions_files/figure-pdf/unnamed-chunk-12-1.pdf}

}

\end{figure}

Prey increase until predators are sufficiently abundant to control them,
at which point they decline, leading to the decline of preadtors. The
predator peaks lag behind the prey peaks

\hypertarget{lotka-volterra-model-4}{%
\subsection{Lotka-Volterra model}\label{lotka-volterra-model-4}}

\begin{itemize}
\tightlist
\item
  A cyclic model: phase plot
\end{itemize}

\begin{Shaded}
\begin{Highlighting}[]
\NormalTok{lv\_results }\SpecialCharTok{\%\textgreater{}\%} 
  \FunctionTok{data.frame}\NormalTok{() }\SpecialCharTok{\%\textgreater{}\%} 
  \FunctionTok{rename}\NormalTok{(}\AttributeTok{prey =}\NormalTok{ x, }\AttributeTok{predator =}\NormalTok{ y) }\SpecialCharTok{\%\textgreater{}\%} 
  \FunctionTok{ggplot}\NormalTok{(}\FunctionTok{aes}\NormalTok{(}\AttributeTok{x =}\NormalTok{ prey, }\AttributeTok{y =}\NormalTok{ predator, }\AttributeTok{color =}\NormalTok{ time)) }\SpecialCharTok{+}
    \FunctionTok{geom\_point}\NormalTok{() }\SpecialCharTok{+}
    \FunctionTok{labs}\NormalTok{(}\AttributeTok{subtitle =} \FunctionTok{paste}\NormalTok{(}\FunctionTok{names}\NormalTok{(pars), pars, }\AttributeTok{sep =} \StringTok{" = "}\NormalTok{, }\AttributeTok{collapse =} \StringTok{"; "}\NormalTok{),}
         \AttributeTok{x =} \StringTok{"Prey"}\NormalTok{, }\AttributeTok{y =} \StringTok{"Predator"}\NormalTok{) }\SpecialCharTok{+} 
    \FunctionTok{theme}\NormalTok{(}\AttributeTok{legend.position =} \StringTok{"bottom"}\NormalTok{,}
          \AttributeTok{aspect.ratio =} \DecValTok{1}\SpecialCharTok{/}\DecValTok{2}\NormalTok{)}
\end{Highlighting}
\end{Shaded}

\begin{figure}[H]

{\centering \includegraphics{ecological_interactions_files/figure-pdf/unnamed-chunk-13-1.pdf}

}

\end{figure}

\hypertarget{a-famous-example}{%
\subsection{A famous example}\label{a-famous-example}}

\begin{itemize}
\tightlist
\item
  Lynx and hare data compiled by Elton\footnote{Elton 1927}
\end{itemize}

\begin{figure}

{\centering \includegraphics{images/lynx_hare.png}

}

\end{figure}

Changes in the abundance of Lynx (Lynx canadensis) {[}top{]} and various
prey species such as Snowshoe Hare (Lepus americanus) are documented in
records of the numbers of pelts brought into the Hudson Bay Company.
Note that the Lynx cycle peaks slightly later than the hare cycle. This
suggests a hypothesis that an abundance of hares feeds an increase in
the number of lynx, which in turn depresses the number of hare and so
on. {[}Unfortunately, hares also cycle in the absence of lynx{]}. The
original data set was compiled by Charles Elton, and is notable as one
of the longest time series in ecology.

\hypertarget{other-predator-prey-models}{%
\subsection{Other predator-prey
models}\label{other-predator-prey-models}}

\begin{itemize}
\tightlist
\item
  How can we incorporate more realism into the Lotka-Volterra model?
\end{itemize}

predator functional response, prey growth rate (numerical response),
predation depending on no. predators?

\hypertarget{functional-respose}{%
\subsection{Functional respose}\label{functional-respose}}

\begin{itemize}
\tightlist
\item
  ``Kill rate'', in the LV model given by:
\end{itemize}

\[\beta N\]

\begin{itemize}
\tightlist
\item
  We call this a Type I functional response
\end{itemize}

\begin{Shaded}
\begin{Highlighting}[]
\NormalTok{n }\OtherTok{\textless{}{-}} \DecValTok{0}\SpecialCharTok{:}\DecValTok{50}
\NormalTok{beta }\OtherTok{\textless{}{-}} \FloatTok{0.2}
\NormalTok{delta }\OtherTok{\textless{}{-}} \FloatTok{0.5}

\NormalTok{d }\OtherTok{\textless{}{-}} \FunctionTok{data.frame}\NormalTok{(}\AttributeTok{N =}\NormalTok{ n) }\SpecialCharTok{\%\textgreater{}\%} 
  \FunctionTok{mutate}\NormalTok{(}\AttributeTok{predation =}\NormalTok{ beta}\SpecialCharTok{*}\NormalTok{delta}\SpecialCharTok{*}\NormalTok{N)}

\FunctionTok{ggplot}\NormalTok{(d, }\FunctionTok{aes}\NormalTok{(N, predation)) }\SpecialCharTok{+}
  \FunctionTok{geom\_line}\NormalTok{(}\AttributeTok{size =} \FloatTok{1.2}\NormalTok{) }\SpecialCharTok{+} 
  \FunctionTok{labs}\NormalTok{(}\AttributeTok{y =} \StringTok{"Predation rate"}\NormalTok{, }\AttributeTok{x =} \StringTok{"Prey population size (N)"}\NormalTok{) }\SpecialCharTok{+}
  \FunctionTok{coord\_cartesian}\NormalTok{(}\AttributeTok{expand =} \DecValTok{0}\NormalTok{) }\SpecialCharTok{+} 
  \FunctionTok{theme}\NormalTok{(}\AttributeTok{aspect.ratio =} \DecValTok{1}\SpecialCharTok{/}\DecValTok{2}\NormalTok{)}
\end{Highlighting}
\end{Shaded}

\begin{figure}[H]

{\centering \includegraphics{ecological_interactions_files/figure-pdf/unnamed-chunk-14-1.pdf}

}

\end{figure}

\hypertarget{other-functional-responses}{%
\subsection{Other functional
responses!}\label{other-functional-responses}}

\begin{itemize}
\item
  Type I: \(\alpha N\)
\item
  Type II: \(\frac {\alpha N}{1+\alpha h N}\)
\item
  Type III: \(\frac {\alpha N^c}{1+\alpha h N^c}\)
\end{itemize}

\hypertarget{other-functional-responses-1}{%
\subsection{Other functional
responses!}\label{other-functional-responses-1}}

\begin{Shaded}
\begin{Highlighting}[]
\NormalTok{n }\OtherTok{\textless{}{-}} \FunctionTok{seq}\NormalTok{(}\DecValTok{0}\NormalTok{, }\DecValTok{2}\NormalTok{, }\FloatTok{0.1}\NormalTok{); a }\OtherTok{\textless{}{-}} \FloatTok{0.5}\NormalTok{; b }\OtherTok{\textless{}{-}} \DecValTok{1}\NormalTok{; c }\OtherTok{\textless{}{-}} \DecValTok{3}

\NormalTok{d }\OtherTok{\textless{}{-}} \FunctionTok{data.frame}\NormalTok{(}\AttributeTok{N =}\NormalTok{ n) }\SpecialCharTok{\%\textgreater{}\%} 
  \FunctionTok{mutate}\NormalTok{(}\StringTok{"Type I"} \OtherTok{=}\NormalTok{ a}\SpecialCharTok{*}\NormalTok{N,}
         \StringTok{"Type II"} \OtherTok{=}\NormalTok{ N }\SpecialCharTok{/}\NormalTok{ (b }\SpecialCharTok{+}\NormalTok{ N),}
         \StringTok{"Type III"} \OtherTok{=}\NormalTok{ N}\SpecialCharTok{\^{}}\NormalTok{c }\SpecialCharTok{/}\NormalTok{ (b }\SpecialCharTok{+}\NormalTok{ N}\SpecialCharTok{\^{}}\NormalTok{c)) }\SpecialCharTok{\%\textgreater{}\%} 
  \FunctionTok{pivot\_longer}\NormalTok{(}\DecValTok{2}\SpecialCharTok{:}\DecValTok{4}\NormalTok{)}
\end{Highlighting}
\end{Shaded}

\begin{figure}

{\centering \includegraphics{ecological_interactions_files/figure-pdf/unnamed-chunk-16-1.pdf}

}

\end{figure}

a is attack rate, and h is handling time. We refer to this functional
response as a type II functional response (Fig. 8.6). As N gets really
big, this fraction approaches 1/h,

When is which FR reasonable?

Ecologists refer to this activity as ``handling'' and the time required
to do that as handling time. Buzz Holling (1959) developed a
mathematical form for a functional response with prey-dependent
predation that includes handling time:

\hypertarget{other-numerical-responses}{%
\subsection{Other numerical
responses!}\label{other-numerical-responses}}

\begin{itemize}
\tightlist
\item
  I.e., the how the predators population growth rate varies with N, or
  dP/dt vs.~N.
\end{itemize}

\hypertarget{other-predator-prey-models-1}{%
\subsection{Other predator-prey
models}\label{other-predator-prey-models-1}}

\begin{itemize}
\item
  Endless possibilities
\item
  Let's look at the Rosenzweig and MacArthur model
\end{itemize}

Rosenzweig \& MacArthur (1963) American Naturalist

\hypertarget{rosenzweig-and-macarthur}{%
\subsection{Rosenzweig and MacArthur}\label{rosenzweig-and-macarthur}}

\[\frac{dN}{dt}=r N(1-\alpha N) - \frac{\alpha N}{1+\alpha h N}P\]

\[\frac{dP}{dt}=e\frac{\alpha N}{1+\alpha h N}P -m P\]

What's going on here? Negative carrying capacity and type II

\hypertarget{rosenzweig-and-macarthur-1}{%
\subsection{Rosenzweig and MacArthur}\label{rosenzweig-and-macarthur-1}}

\begin{Shaded}
\begin{Highlighting}[]
\CommentTok{\# https://hankstevens.github.io/Primer{-}of{-}Ecology/cr.html}
\NormalTok{cr\_RM\_pred }\OtherTok{\textless{}{-}} \ControlFlowTok{function}\NormalTok{(time, y, p)\{}
\NormalTok{  N }\OtherTok{\textless{}{-}}\NormalTok{ y[}\DecValTok{1}\NormalTok{]}
\NormalTok{  P }\OtherTok{\textless{}{-}}\NormalTok{ y[}\DecValTok{2}\NormalTok{]}
  \FunctionTok{with}\NormalTok{(}\FunctionTok{as.list}\NormalTok{(p), \{}
\NormalTok{    Ndot }\OtherTok{\textless{}{-}}\NormalTok{ r}\SpecialCharTok{*}\NormalTok{N}\SpecialCharTok{*}\NormalTok{(}\DecValTok{1} \SpecialCharTok{{-}}\NormalTok{ alpha}\SpecialCharTok{*}\NormalTok{N) }\SpecialCharTok{{-}}\NormalTok{ a}\SpecialCharTok{*}\NormalTok{N}\SpecialCharTok{*}\NormalTok{P}\SpecialCharTok{/}\NormalTok{(}\DecValTok{1} \SpecialCharTok{+}\NormalTok{ a}\SpecialCharTok{*}\NormalTok{h}\SpecialCharTok{*}\NormalTok{N) }\CommentTok{\# prey}
\NormalTok{    Pdot }\OtherTok{\textless{}{-}}\NormalTok{ e}\SpecialCharTok{*}\NormalTok{a}\SpecialCharTok{*}\NormalTok{N}\SpecialCharTok{*}\NormalTok{P}\SpecialCharTok{/}\NormalTok{(}\DecValTok{1} \SpecialCharTok{+}\NormalTok{ a}\SpecialCharTok{*}\NormalTok{h}\SpecialCharTok{*}\NormalTok{N) }\SpecialCharTok{{-}}\NormalTok{ m}\SpecialCharTok{*}\NormalTok{P }\CommentTok{\# predator}
    \FunctionTok{return}\NormalTok{(}\FunctionTok{list}\NormalTok{(}\FunctionTok{c}\NormalTok{(Ndot, Pdot)))}
\NormalTok{  \})}
\NormalTok{\}}

\NormalTok{t }\OtherTok{\textless{}{-}} \DecValTok{0}\SpecialCharTok{:}\DecValTok{200}
\NormalTok{y0 }\OtherTok{\textless{}{-}} \FunctionTok{c}\NormalTok{(}\AttributeTok{N =} \DecValTok{100}\NormalTok{, }\AttributeTok{P =} \DecValTok{3}\NormalTok{)}

\NormalTok{p }\OtherTok{\textless{}{-}} \FunctionTok{list}\NormalTok{(}\AttributeTok{r =} \FloatTok{0.8}\NormalTok{, }\AttributeTok{alpha =} \FloatTok{0.001}\NormalTok{, }\AttributeTok{a =} \FloatTok{0.02}\NormalTok{, }\AttributeTok{e =} \FloatTok{0.04}\NormalTok{, }\AttributeTok{m =}\NormalTok{ .}\DecValTok{15}\NormalTok{, }\AttributeTok{h =}\NormalTok{ .}\DecValTok{1}\NormalTok{)}

\NormalTok{outdf }\OtherTok{\textless{}{-}} \FunctionTok{as.data.frame}\NormalTok{( }\FunctionTok{ode}\NormalTok{(y0, t, cr\_RM\_pred,  p) ) }\SpecialCharTok{\%\textgreater{}\%}
  \FunctionTok{pivot\_longer}\NormalTok{(}\SpecialCharTok{{-}}\NormalTok{time, }\AttributeTok{names\_to =} \StringTok{"Population"}\NormalTok{, }\AttributeTok{values\_to =} \StringTok{"Abundance"}\NormalTok{)}
\end{Highlighting}
\end{Shaded}

\hypertarget{rosenzweig-and-macarthur-2}{%
\subsection{Rosenzweig and MacArthur}\label{rosenzweig-and-macarthur-2}}

\begin{Shaded}
\begin{Highlighting}[]
\FunctionTok{ggplot}\NormalTok{(outdf, }\FunctionTok{aes}\NormalTok{(time, Abundance, }\AttributeTok{colour =}\NormalTok{ Population)) }\SpecialCharTok{+}
  \FunctionTok{geom\_line}\NormalTok{(}\AttributeTok{size =} \FloatTok{1.2}\NormalTok{) }\SpecialCharTok{+} 
  \FunctionTok{scale\_color\_brewer}\NormalTok{(}\AttributeTok{palette =} \StringTok{"Set2"}\NormalTok{) }\SpecialCharTok{+}
  \FunctionTok{labs}\NormalTok{(}\AttributeTok{x =} \StringTok{"Time"}\NormalTok{)}
\end{Highlighting}
\end{Shaded}

\begin{figure}[H]

{\centering \includegraphics{ecological_interactions_files/figure-pdf/unnamed-chunk-18-1.pdf}

}

\end{figure}

A damped oscillation means an oscillation that fades away with time.
Eventually we will reach an equilibrium

\hypertarget{rosenzweig-and-macarthur-3}{%
\subsection{Rosenzweig and MacArthur}\label{rosenzweig-and-macarthur-3}}

\begin{itemize}
\tightlist
\item
  Paradox of enrichment!
\end{itemize}

\begin{figure}

{\centering \includegraphics{images/paradox.png}

}

\end{figure}

In each case, increasing the supply of limiting nutrients or energy
tends to destroy the steady state. Thus man must be very careful in
attempting to enrich an ecosystem in order to increase its food yield.
There is a real chance that such activity may result in decimation of
the food species that are wanted in greater abundance.

\hypertarget{other-types-of-cycles}{%
\subsection{Other types of cycles\ldots{}}\label{other-types-of-cycles}}

\begin{itemize}
\tightlist
\item
  We have seen predator-prey cycles, but populations can also cycle for
  other reasons!
\end{itemize}

\hypertarget{cohort-cycles}{%
\subsection{Cohort-cycles}\label{cohort-cycles}}

\begin{figure}

{\centering \includegraphics{images/cohort_cycles.png}

}

\end{figure}

de Roos \emph{et al} (2003) Ecology Letters

\hypertarget{competition}{%
\subsection{Competition}\label{competition}}

\begin{itemize}
\item
  Interference
\item
  Scramble / Exploitation
\end{itemize}

\begin{figure}

\begin{minipage}[t]{0.50\linewidth}

{\centering 

\raisebox{-\height}{

\includegraphics{images/scramble.png}

}

}

\end{minipage}%
%
\begin{minipage}[t]{0.50\linewidth}

{\centering 

\raisebox{-\height}{

\includegraphics{images/deer.png}

}

}

\end{minipage}%

\end{figure}

Which one have we already talked about?

Male-male competition in red deer during rut is an example of
interference competition within a species.

Scramble competition, occurs indirectly when organisms both use a common
limiting resource or shared food item. Instead of fighting or exhibiting
aggressive behavior in order to win resources, exploitative competition
occurs when resource use by one organism depletes the total amount
available for other organism

\hypertarget{l-v-competition-model}{%
\subsection{L-V competition model}\label{l-v-competition-model}}

\[\frac{dN_1}{dt}=r_1N_1(\frac{K_1-N_1-\alpha_{12}N_2}{K_1})\]
\[\frac{dN_2}{dt}=r_2N_1(\frac{K_2-N_2-\alpha_{21}N_1}{K_2})\]

alpha12 expresses the effect of one member of Population 2 on the growth
rate of Population 1 alpha21 expresses the effect of one member of
Population 1 on the growth rate of Population 2.

\hypertarget{l-v-competition-model-1}{%
\subsection{L-V competition model}\label{l-v-competition-model-1}}

\begin{itemize}
\tightlist
\item
  Find the zero-growth isocline for species \(N_1\)
\end{itemize}

\[\frac{dN_1}{dt}=r_1N_1(\frac{K_1-N_1-\alpha_{12}N_2}{K_1})\]
\[0=r_1N_1(\frac{K_1-N_1-\alpha_{12}N_2}{K_1})\]

\[0=\frac{K_1-N_1-\alpha_{12}N_2}{K_1}\]

\[N_1=K_1-\alpha_{12}N_2\]

The equilibrium is the population size at which the population stops
growing. We can solve for this equilibrium by setting dN/dt for each
species to zero and solving for N.

\hypertarget{l-v-competition-model-2}{%
\subsection{L-V competition model}\label{l-v-competition-model-2}}

\begin{itemize}
\tightlist
\item
  Find the zero-growth isocline for species \(N_2\)
\end{itemize}

\[\frac{dN_2}{dt}=r_2N_1(\frac{K_2-N_2-\alpha_{21}N_1}{K_2})\]

\[N_2=K_2-\alpha_{21}N_1\]

Linear equation! We can use N1 and N2 as coordinates to graph each
species ZNGI. We can graph this isocline by finding any two points along
it and connecting them with a straight line.

\hypertarget{l-v-competition-model-3}{%
\subsection{L-V competition model}\label{l-v-competition-model-3}}

\begin{Shaded}
\begin{Highlighting}[]
\NormalTok{N }\OtherTok{\textless{}{-}} \DecValTok{0}\SpecialCharTok{:}\DecValTok{20}

\CommentTok{\# Species 1 wins}
\NormalTok{a\_21 }\OtherTok{\textless{}{-}} \FloatTok{0.75}
\NormalTok{a\_12 }\OtherTok{\textless{}{-}} \DecValTok{1}
\NormalTok{K\_1 }\OtherTok{\textless{}{-}} \FloatTok{1.5}
\NormalTok{K\_2 }\OtherTok{\textless{}{-}} \DecValTok{1}

\NormalTok{d1 }\OtherTok{\textless{}{-}} \FunctionTok{data.frame}\NormalTok{(}\AttributeTok{N =}\NormalTok{ N) }\SpecialCharTok{\%\textgreater{}\%} 
  \FunctionTok{mutate}\NormalTok{(}\AttributeTok{N2\_iso =}\NormalTok{ K\_2 }\SpecialCharTok{{-}}\NormalTok{ a\_21}\SpecialCharTok{*}\NormalTok{N,}
         \AttributeTok{N1\_iso =}\NormalTok{ K\_1 }\SpecialCharTok{{-}}\NormalTok{ a\_12}\SpecialCharTok{*}\NormalTok{N,}
         \AttributeTok{scenario =} \StringTok{"Species 1 wins"}\NormalTok{)}
\end{Highlighting}
\end{Shaded}

\hypertarget{l-v-competition-model-4}{%
\subsection{L-V competition model}\label{l-v-competition-model-4}}

\begin{Shaded}
\begin{Highlighting}[]
\NormalTok{p1 }\OtherTok{\textless{}{-}} \FunctionTok{ggplot}\NormalTok{(}\AttributeTok{data =}\NormalTok{ d1) }\SpecialCharTok{+}
  \FunctionTok{geom\_line}\NormalTok{(}\FunctionTok{aes}\NormalTok{(}\AttributeTok{x =}\NormalTok{ N, }\AttributeTok{y =}\NormalTok{ N2\_iso, }\AttributeTok{color =} \StringTok{"N2\_iso"}\NormalTok{), }\AttributeTok{size =} \FloatTok{1.2}\NormalTok{) }\SpecialCharTok{+}
  \FunctionTok{geom\_line}\NormalTok{(}\FunctionTok{aes}\NormalTok{(}\AttributeTok{x =}\NormalTok{ N, }\AttributeTok{y =}\NormalTok{ N1\_iso, }\AttributeTok{color =} \StringTok{"N1\_iso"}\NormalTok{), }\AttributeTok{size =} \FloatTok{1.2}\NormalTok{) }\SpecialCharTok{+}
  \FunctionTok{coord\_cartesian}\NormalTok{(}\AttributeTok{ylim =} \FunctionTok{c}\NormalTok{(}\DecValTok{0}\NormalTok{, }\FloatTok{1.5}\NormalTok{), }\AttributeTok{xlim =} \FunctionTok{c}\NormalTok{(}\DecValTok{0}\NormalTok{, }\FloatTok{1.5}\NormalTok{), }\AttributeTok{expand =} \DecValTok{0}\NormalTok{) }\SpecialCharTok{+}
  \FunctionTok{labs}\NormalTok{(}\AttributeTok{y =} \StringTok{"Species 2"}\NormalTok{, }\AttributeTok{x =} \StringTok{"Species 1"}\NormalTok{) }\SpecialCharTok{+}
  \ConstantTok{NULL}

\CommentTok{\# Species 2 wins}
\NormalTok{a\_21 }\OtherTok{\textless{}{-}} \DecValTok{1}
\NormalTok{a\_12 }\OtherTok{\textless{}{-}} \FloatTok{0.75}
\NormalTok{K\_1 }\OtherTok{\textless{}{-}} \DecValTok{1}
\NormalTok{K\_2 }\OtherTok{\textless{}{-}} \FloatTok{1.5}

\NormalTok{d2 }\OtherTok{\textless{}{-}} \FunctionTok{data.frame}\NormalTok{(}\AttributeTok{N =}\NormalTok{ N) }\SpecialCharTok{\%\textgreater{}\%} 
  \FunctionTok{mutate}\NormalTok{(}\AttributeTok{N2\_iso =}\NormalTok{ K\_2 }\SpecialCharTok{{-}}\NormalTok{ a\_21}\SpecialCharTok{*}\NormalTok{N,}
         \AttributeTok{N1\_iso =}\NormalTok{ K\_1 }\SpecialCharTok{{-}}\NormalTok{ a\_12}\SpecialCharTok{*}\NormalTok{N,}
         \AttributeTok{scenario =} \StringTok{"Species 2 wins"}\NormalTok{)}

\NormalTok{p2 }\OtherTok{\textless{}{-}} \FunctionTok{ggplot}\NormalTok{(}\AttributeTok{data =}\NormalTok{ d2) }\SpecialCharTok{+}
  \FunctionTok{geom\_line}\NormalTok{(}\FunctionTok{aes}\NormalTok{(}\AttributeTok{x =}\NormalTok{ N, }\AttributeTok{y =}\NormalTok{ N2\_iso, }\AttributeTok{color =} \StringTok{"N2\_iso"}\NormalTok{), }\AttributeTok{size =} \FloatTok{1.2}\NormalTok{) }\SpecialCharTok{+}
  \FunctionTok{geom\_line}\NormalTok{(}\FunctionTok{aes}\NormalTok{(}\AttributeTok{x =}\NormalTok{ N, }\AttributeTok{y =}\NormalTok{ N1\_iso, }\AttributeTok{color =} \StringTok{"N1\_iso"}\NormalTok{), }\AttributeTok{size =} \FloatTok{1.2}\NormalTok{) }\SpecialCharTok{+}
  \FunctionTok{coord\_cartesian}\NormalTok{(}\AttributeTok{ylim =} \FunctionTok{c}\NormalTok{(}\DecValTok{0}\NormalTok{, }\FloatTok{1.5}\NormalTok{), }\AttributeTok{xlim =} \FunctionTok{c}\NormalTok{(}\DecValTok{0}\NormalTok{, }\FloatTok{1.5}\NormalTok{), }\AttributeTok{expand =} \DecValTok{0}\NormalTok{) }\SpecialCharTok{+}
  \FunctionTok{labs}\NormalTok{(}\AttributeTok{y =} \StringTok{"Species 2"}\NormalTok{, }\AttributeTok{x =} \StringTok{"Species 1"}\NormalTok{) }\SpecialCharTok{+}
  \ConstantTok{NULL}

\CommentTok{\# Unstable equilibrium}
\NormalTok{a\_21 }\OtherTok{\textless{}{-}} \FloatTok{1.25}
\NormalTok{a\_12 }\OtherTok{\textless{}{-}} \FloatTok{0.75}
\NormalTok{K\_1 }\OtherTok{\textless{}{-}} \DecValTok{1}
\NormalTok{K\_2 }\OtherTok{\textless{}{-}} \FloatTok{1.5}

\NormalTok{d3 }\OtherTok{\textless{}{-}} \FunctionTok{data.frame}\NormalTok{(}\AttributeTok{N =}\NormalTok{ N) }\SpecialCharTok{\%\textgreater{}\%} 
  \FunctionTok{mutate}\NormalTok{(}\AttributeTok{N2\_iso =}\NormalTok{ K\_2 }\SpecialCharTok{{-}}\NormalTok{ a\_21}\SpecialCharTok{*}\NormalTok{N,}
         \AttributeTok{N1\_iso =}\NormalTok{ K\_1 }\SpecialCharTok{{-}}\NormalTok{ a\_12}\SpecialCharTok{*}\NormalTok{N,}
         \AttributeTok{scenario =} \StringTok{"Unstable equilibrium"}\NormalTok{)}

\NormalTok{p3 }\OtherTok{\textless{}{-}} \FunctionTok{ggplot}\NormalTok{(}\AttributeTok{data =}\NormalTok{ d3) }\SpecialCharTok{+}
  \FunctionTok{geom\_line}\NormalTok{(}\FunctionTok{aes}\NormalTok{(}\AttributeTok{x =}\NormalTok{ N, }\AttributeTok{y =}\NormalTok{ N2\_iso, }\AttributeTok{color =} \StringTok{"N2\_iso"}\NormalTok{), }\AttributeTok{size =} \FloatTok{1.2}\NormalTok{) }\SpecialCharTok{+}
  \FunctionTok{geom\_line}\NormalTok{(}\FunctionTok{aes}\NormalTok{(}\AttributeTok{x =}\NormalTok{ N, }\AttributeTok{y =}\NormalTok{ N1\_iso, }\AttributeTok{color =} \StringTok{"N1\_iso"}\NormalTok{), }\AttributeTok{size =} \FloatTok{1.2}\NormalTok{) }\SpecialCharTok{+}
  \FunctionTok{coord\_cartesian}\NormalTok{(}\AttributeTok{ylim =} \FunctionTok{c}\NormalTok{(}\DecValTok{0}\NormalTok{, }\FloatTok{1.5}\NormalTok{), }\AttributeTok{xlim =} \FunctionTok{c}\NormalTok{(}\DecValTok{0}\NormalTok{, }\FloatTok{1.5}\NormalTok{), }\AttributeTok{expand =} \DecValTok{0}\NormalTok{) }\SpecialCharTok{+}
  \FunctionTok{labs}\NormalTok{(}\AttributeTok{y =} \StringTok{"Species 2"}\NormalTok{, }\AttributeTok{x =} \StringTok{"Species 1"}\NormalTok{) }\SpecialCharTok{+}
  \ConstantTok{NULL}

\CommentTok{\# Stable equilibrium}
\NormalTok{a\_21 }\OtherTok{\textless{}{-}} \FloatTok{0.5}
\NormalTok{a\_12 }\OtherTok{\textless{}{-}} \DecValTok{1}
\NormalTok{K\_1 }\OtherTok{\textless{}{-}} \FloatTok{1.5}
\NormalTok{K\_2 }\OtherTok{\textless{}{-}} \DecValTok{1}

\NormalTok{d4 }\OtherTok{\textless{}{-}} \FunctionTok{data.frame}\NormalTok{(}\AttributeTok{N =}\NormalTok{ N) }\SpecialCharTok{\%\textgreater{}\%} 
  \FunctionTok{mutate}\NormalTok{(}\AttributeTok{N2\_iso =}\NormalTok{ K\_2 }\SpecialCharTok{{-}}\NormalTok{ a\_21}\SpecialCharTok{*}\NormalTok{N,}
         \AttributeTok{N1\_iso =}\NormalTok{ K\_1 }\SpecialCharTok{{-}}\NormalTok{ a\_12}\SpecialCharTok{*}\NormalTok{N,}
         \AttributeTok{scenario =} \StringTok{"Stable equilibrium"}\NormalTok{)}

\NormalTok{p4 }\OtherTok{\textless{}{-}} \FunctionTok{ggplot}\NormalTok{(}\AttributeTok{data =}\NormalTok{ d) }\SpecialCharTok{+}
  \FunctionTok{geom\_line}\NormalTok{(}\FunctionTok{aes}\NormalTok{(}\AttributeTok{x =}\NormalTok{ N, }\AttributeTok{y =}\NormalTok{ N2\_iso, }\AttributeTok{color =} \StringTok{"N2\_iso"}\NormalTok{), }\AttributeTok{size =} \FloatTok{1.2}\NormalTok{) }\SpecialCharTok{+}
  \FunctionTok{geom\_line}\NormalTok{(}\FunctionTok{aes}\NormalTok{(}\AttributeTok{x =}\NormalTok{ N, }\AttributeTok{y =}\NormalTok{ N1\_iso, }\AttributeTok{color =} \StringTok{"N1\_iso"}\NormalTok{), }\AttributeTok{size =} \FloatTok{1.2}\NormalTok{) }\SpecialCharTok{+}
  \FunctionTok{coord\_cartesian}\NormalTok{(}\AttributeTok{ylim =} \FunctionTok{c}\NormalTok{(}\DecValTok{0}\NormalTok{, }\FloatTok{1.5}\NormalTok{), }\AttributeTok{xlim =} \FunctionTok{c}\NormalTok{(}\DecValTok{0}\NormalTok{, }\FloatTok{1.5}\NormalTok{), }\AttributeTok{expand =} \DecValTok{0}\NormalTok{) }\SpecialCharTok{+}
  \FunctionTok{labs}\NormalTok{(}\AttributeTok{y =} \StringTok{"Species 2"}\NormalTok{, }\AttributeTok{x =} \StringTok{"Species 1"}\NormalTok{) }\SpecialCharTok{+}
  \ConstantTok{NULL}

\CommentTok{\#(p1 | p2) /  (p3 | p4) + plot\_layout(guides = "collect")}

\NormalTok{d }\OtherTok{\textless{}{-}} \FunctionTok{bind\_rows}\NormalTok{(d1, d2, d3, d4)}

\FunctionTok{ggplot}\NormalTok{(}\AttributeTok{data =}\NormalTok{ d) }\SpecialCharTok{+} 
  \FunctionTok{geom\_line}\NormalTok{(}\FunctionTok{aes}\NormalTok{(}\AttributeTok{x =}\NormalTok{ N, }\AttributeTok{y =}\NormalTok{ N2\_iso, }\AttributeTok{color =} \StringTok{"2 isocline"}\NormalTok{), }\AttributeTok{size =} \FloatTok{1.2}\NormalTok{) }\SpecialCharTok{+}
  \FunctionTok{geom\_line}\NormalTok{(}\FunctionTok{aes}\NormalTok{(}\AttributeTok{x =}\NormalTok{ N, }\AttributeTok{y =}\NormalTok{ N1\_iso, }\AttributeTok{color =} \StringTok{"1 isocline"}\NormalTok{), }\AttributeTok{size =} \FloatTok{1.2}\NormalTok{) }\SpecialCharTok{+}
  \FunctionTok{coord\_cartesian}\NormalTok{(}\AttributeTok{ylim =} \FunctionTok{c}\NormalTok{(}\DecValTok{0}\NormalTok{, }\FloatTok{1.5}\NormalTok{), }\AttributeTok{xlim =} \FunctionTok{c}\NormalTok{(}\DecValTok{0}\NormalTok{, }\FloatTok{1.5}\NormalTok{), }\AttributeTok{expand =} \DecValTok{0}\NormalTok{) }\SpecialCharTok{+}
  \FunctionTok{labs}\NormalTok{(}\AttributeTok{y =} \StringTok{"Species 2"}\NormalTok{, }\AttributeTok{x =} \StringTok{"Species 1"}\NormalTok{, }\AttributeTok{color =} \StringTok{"Species"}\NormalTok{) }\SpecialCharTok{+}
  \FunctionTok{facet\_wrap}\NormalTok{(}\SpecialCharTok{\textasciitilde{}}\NormalTok{scenario, }\AttributeTok{ncol =} \DecValTok{2}\NormalTok{, }\AttributeTok{scales =} \StringTok{"free"}\NormalTok{) }\SpecialCharTok{+} 
  \FunctionTok{scale\_color\_brewer}\NormalTok{(}\AttributeTok{palette =} \StringTok{"Set2"}\NormalTok{) }\SpecialCharTok{+}
  \FunctionTok{theme}\NormalTok{(}\AttributeTok{aspect.ratio =} \DecValTok{4}\SpecialCharTok{/}\DecValTok{5}\NormalTok{,}
        \AttributeTok{strip.background =} \FunctionTok{element\_rect}\NormalTok{(}\AttributeTok{fill =} \StringTok{"gray95"}\NormalTok{),}
        \AttributeTok{strip.text =} \FunctionTok{element\_text}\NormalTok{(}\AttributeTok{margin =} \FunctionTok{margin}\NormalTok{(}\AttributeTok{b =} \DecValTok{2}\NormalTok{, }\AttributeTok{t =} \DecValTok{2}\NormalTok{), }\AttributeTok{color =} \StringTok{"grey10"}\NormalTok{))}
\end{Highlighting}
\end{Shaded}

\begin{figure}[H]

{\centering \includegraphics{ecological_interactions_files/figure-pdf/unnamed-chunk-20-1.pdf}

}

\end{figure}

Simply put, for species to coexist stably, their effects on themselves
must be greater than their effects on each other

\hypertarget{epilogue}{%
\subsection{Epilogue}\label{epilogue}}

\begin{itemize}
\tightlist
\item
  If models keep being still are too unrealistic even after adding
  realistic features, why do we use them?
\end{itemize}

Also, what do we not include here? Stochasticity, environmental
dependence Interactions vary with space and time and environment and
size\ldots{}

\begin{figure}

{\centering \includegraphics{images/g_box.png}

}

\end{figure}

Also, what do we not include here? Stochasticity, environmental
dependence Interactions vary with space and time and environment and
size\ldots{}

\hypertarget{quantifying-species-interactions-in-natural-systems}{%
\section{3. Quantifying species interactions in natural
systems}\label{quantifying-species-interactions-in-natural-systems}}

\hypertarget{quantifying-species-interactions-in-natural-systems-1}{%
\subsection{3. Quantifying species interactions in natural
systems}\label{quantifying-species-interactions-in-natural-systems-1}}

\begin{itemize}
\item
  We know how some ideas on how to model interactions mathematically
\item
  How do we determine if and what species interactions occur in the
  wild?
\end{itemize}

Diet data, trait-data, Abundance data (e.g., from scientific trawl
surveys), Apply statistical models

\hypertarget{diet-data}{%
\subsection{Diet data}\label{diet-data}}

\begin{itemize}
\tightlist
\item
  We collect stomach content data for key species (read: commercial)
\end{itemize}

\begin{figure}

{\centering \includegraphics{images/diet.png}

}

\end{figure}

\hypertarget{diet-data-example-baltic-cod}{%
\subsection{Diet data: example Baltic
cod}\label{diet-data-example-baltic-cod}}

\hypertarget{diet-data-example-baltic-cod-1}{%
\subsection{Diet data: example Baltic
cod}\label{diet-data-example-baltic-cod-1}}

\begin{itemize}
\tightlist
\item
  Bioenergetic models + diet information = cod grow worse today
\end{itemize}

\begin{figure}

{\centering \includegraphics[width=0.55\textwidth,height=\textheight]{images/eq1.png}

}

\end{figure}

\begin{figure}

{\centering \includegraphics[width=0.6\textwidth,height=\textheight]{images/eq2.png}

}

\end{figure}

\begin{figure}

{\centering \includegraphics[width=0.5\textwidth,height=\textheight]{images/eq3.png}

}

\end{figure}

Neuenfeldt \emph{et al} (2020) ICESJMS

\hypertarget{diet-data-example-baltic-cod-2}{%
\subsection{Diet data: example Baltic
cod}\label{diet-data-example-baltic-cod-2}}

\begin{itemize}
\tightlist
\item
  Bioenergetic models + diet information = cod grow worse today
\end{itemize}

\begin{figure}

{\centering \includegraphics{images/neuen.png}

}

\end{figure}

Neuenfeldt \emph{et al} (2020) ICESJMS

\hypertarget{diet-data-example-baltic-cod-3}{%
\subsection{Diet data: example Baltic
cod}\label{diet-data-example-baltic-cod-3}}

\begin{figure}

{\centering \includegraphics{images/skinny_cod.png}

}

\end{figure}

Lunneryd 2020, Aqua Reports

\hypertarget{predation-in-space}{%
\subsection{Predation in space!}\label{predation-in-space}}

\begin{itemize}
\tightlist
\item
  Diets, predators, and prey vary in space
\end{itemize}

\begin{figure}

{\centering \includegraphics{images/goodman1.png}

}

\end{figure}

Goodman \emph{et al} (2022) Ecography

This paper also uses biomass data\ldots{}

\hypertarget{predation-in-space-1}{%
\subsection{Predation in space!}\label{predation-in-space-1}}

\begin{itemize}
\tightlist
\item
  Diets, predators, and prey vary in space
\end{itemize}

\begin{figure}

{\centering \includegraphics{images/goodman2.png}

}

\end{figure}

Goodman \emph{et al} (2022) Ecography

\hypertarget{predation-in-space-2}{%
\subsection{Predation in space!}\label{predation-in-space-2}}

\begin{itemize}
\tightlist
\item
  Diets, predators, and prey vary in space
\end{itemize}

\begin{figure}

{\centering \includegraphics{images/goodman3.png}

}

\end{figure}

Goodman \emph{et al} (2022) Ecography

\hypertarget{competition-1}{%
\subsection{Competition}\label{competition-1}}

\begin{itemize}
\item
  Sprat and herring affect the abundance of zooplankton
\item
  In sprat, zooplankton in stomachs correlated with condition
\end{itemize}

\begin{figure}

{\centering \includegraphics{images/casini_sprat.png}

}

\end{figure}

Casini \emph{et al} (2006) Oikos

Here we use in addition some other metric: condition! require other
samples

Also, we haven't touched on experiments

\hypertarget{species-interactions-in-fisheries-research}{%
\section{4. Species interactions in fisheries
research}\label{species-interactions-in-fisheries-research}}

\hypertarget{species-interactions-in-fisheries-research-1}{%
\subsection{4. Species interactions in fisheries
research}\label{species-interactions-in-fisheries-research-1}}

\begin{itemize}
\tightlist
\item
  Statistical or Mathematical approaches
\end{itemize}

\hypertarget{in-stock-assessment}{%
\subsection{In stock assessment}\label{in-stock-assessment}}

\begin{itemize}
\item
  Mainly in terms of predation (natural mortality), not on growth
  parameters (competition!
\item
  SMS model: catch-at-age, survey abundance, stomach data
\item
  (statistical approach)
\end{itemize}

Lewy \& Vinther (2004) ICES CM

\hypertarget{natural-mortality-rates}{%
\subsection{Natural mortality rates}\label{natural-mortality-rates}}

\begin{figure}

{\centering \includegraphics{images/sms.png}

}

\end{figure}

Lewy \& Vinther (2004) ICES CM

\hypertarget{food-web-models-strategic-advice}{%
\subsection{Food web models: strategic
advice}\label{food-web-models-strategic-advice}}

\begin{itemize}
\tightlist
\item
  Size-spectrum models
\end{itemize}

\begin{figure}

{\centering \includegraphics{images/blanch3.png}

}

\end{figure}

Blanchard et al (2017) TREE

Read the boxes

\hypertarget{food-web-models-strategic-advice-1}{%
\subsection{Food web models: strategic
advice}\label{food-web-models-strategic-advice-1}}

\begin{figure}

{\centering \includegraphics{images/concept_figure.png}

}

\end{figure}

Lindmark et al (2022) GCB; Scott et al (2014) Methods in Ecology \&
Evolution; Andersen (2019) Princeton University Press

strategic vs tactical, Based on ``first principles'', higher
independence

\hypertarget{food-web-models-strategic-advice-2}{%
\subsection{Food web models: strategic
advice}\label{food-web-models-strategic-advice-2}}

\begin{figure}

{\centering \includegraphics{images/blanch1.png}

}

\end{figure}

Blanchard et al (2014) Journal of Applied Ecology

strategic vs tactical

\hypertarget{food-web-models-strategic-advice-3}{%
\subsection{Food web models: strategic
advice}\label{food-web-models-strategic-advice-3}}

\begin{figure}

{\centering \includegraphics{images/blanch2.png}

}

\end{figure}

Blanchard et al (2014) Journal of Applied Ecology

strategic vs tactical

\hypertarget{mathematical-or-statistical}{%
\subsection{Mathematical or
statistical?}\label{mathematical-or-statistical}}

\includegraphics{images/whynotboth.png} ::: \{.notes\} Many of these
models are based upon statistical techniques and are good at assessing
the current state and making short-term predictions; however, as they do
not model interactions between stocks, they lack predictive power on
longer timescales. Additionally, there are size-based multi-species
models that represent key biological processes and consider interactions
between stocks such as predation and competition for resources. Due to
the complexity of these models, they are difficult to fit to data, and
so many size-based multi-species models depend upon single-species
models where they exist, or ad hoc assumptions when they do not, for
parameters such as annual fishing mortality. :::

\hypertarget{question-time}{%
\subsection{Question time!}\label{question-time}}



\end{document}
